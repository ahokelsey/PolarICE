\documentclass[]{article}
\usepackage{lmodern}
\usepackage{amssymb,amsmath}
\usepackage{ifxetex,ifluatex}
\usepackage{fixltx2e} % provides \textsubscript
\ifnum 0\ifxetex 1\fi\ifluatex 1\fi=0 % if pdftex
  \usepackage[T1]{fontenc}
  \usepackage[utf8]{inputenc}
\else % if luatex or xelatex
  \ifxetex
    \usepackage{mathspec}
  \else
    \usepackage{fontspec}
  \fi
  \defaultfontfeatures{Ligatures=TeX,Scale=MatchLowercase}
\fi
% use upquote if available, for straight quotes in verbatim environments
\IfFileExists{upquote.sty}{\usepackage{upquote}}{}
% use microtype if available
\IfFileExists{microtype.sty}{%
\usepackage{microtype}
\UseMicrotypeSet[protrusion]{basicmath} % disable protrusion for tt fonts
}{}
\usepackage[margin=1in]{geometry}
\usepackage{hyperref}
\hypersetup{unicode=true,
            pdftitle={R Notebook},
            pdfborder={0 0 0},
            breaklinks=true}
\urlstyle{same}  % don't use monospace font for urls
\usepackage{color}
\usepackage{fancyvrb}
\newcommand{\VerbBar}{|}
\newcommand{\VERB}{\Verb[commandchars=\\\{\}]}
\DefineVerbatimEnvironment{Highlighting}{Verbatim}{commandchars=\\\{\}}
% Add ',fontsize=\small' for more characters per line
\usepackage{framed}
\definecolor{shadecolor}{RGB}{248,248,248}
\newenvironment{Shaded}{\begin{snugshade}}{\end{snugshade}}
\newcommand{\KeywordTok}[1]{\textcolor[rgb]{0.13,0.29,0.53}{\textbf{#1}}}
\newcommand{\DataTypeTok}[1]{\textcolor[rgb]{0.13,0.29,0.53}{#1}}
\newcommand{\DecValTok}[1]{\textcolor[rgb]{0.00,0.00,0.81}{#1}}
\newcommand{\BaseNTok}[1]{\textcolor[rgb]{0.00,0.00,0.81}{#1}}
\newcommand{\FloatTok}[1]{\textcolor[rgb]{0.00,0.00,0.81}{#1}}
\newcommand{\ConstantTok}[1]{\textcolor[rgb]{0.00,0.00,0.00}{#1}}
\newcommand{\CharTok}[1]{\textcolor[rgb]{0.31,0.60,0.02}{#1}}
\newcommand{\SpecialCharTok}[1]{\textcolor[rgb]{0.00,0.00,0.00}{#1}}
\newcommand{\StringTok}[1]{\textcolor[rgb]{0.31,0.60,0.02}{#1}}
\newcommand{\VerbatimStringTok}[1]{\textcolor[rgb]{0.31,0.60,0.02}{#1}}
\newcommand{\SpecialStringTok}[1]{\textcolor[rgb]{0.31,0.60,0.02}{#1}}
\newcommand{\ImportTok}[1]{#1}
\newcommand{\CommentTok}[1]{\textcolor[rgb]{0.56,0.35,0.01}{\textit{#1}}}
\newcommand{\DocumentationTok}[1]{\textcolor[rgb]{0.56,0.35,0.01}{\textbf{\textit{#1}}}}
\newcommand{\AnnotationTok}[1]{\textcolor[rgb]{0.56,0.35,0.01}{\textbf{\textit{#1}}}}
\newcommand{\CommentVarTok}[1]{\textcolor[rgb]{0.56,0.35,0.01}{\textbf{\textit{#1}}}}
\newcommand{\OtherTok}[1]{\textcolor[rgb]{0.56,0.35,0.01}{#1}}
\newcommand{\FunctionTok}[1]{\textcolor[rgb]{0.00,0.00,0.00}{#1}}
\newcommand{\VariableTok}[1]{\textcolor[rgb]{0.00,0.00,0.00}{#1}}
\newcommand{\ControlFlowTok}[1]{\textcolor[rgb]{0.13,0.29,0.53}{\textbf{#1}}}
\newcommand{\OperatorTok}[1]{\textcolor[rgb]{0.81,0.36,0.00}{\textbf{#1}}}
\newcommand{\BuiltInTok}[1]{#1}
\newcommand{\ExtensionTok}[1]{#1}
\newcommand{\PreprocessorTok}[1]{\textcolor[rgb]{0.56,0.35,0.01}{\textit{#1}}}
\newcommand{\AttributeTok}[1]{\textcolor[rgb]{0.77,0.63,0.00}{#1}}
\newcommand{\RegionMarkerTok}[1]{#1}
\newcommand{\InformationTok}[1]{\textcolor[rgb]{0.56,0.35,0.01}{\textbf{\textit{#1}}}}
\newcommand{\WarningTok}[1]{\textcolor[rgb]{0.56,0.35,0.01}{\textbf{\textit{#1}}}}
\newcommand{\AlertTok}[1]{\textcolor[rgb]{0.94,0.16,0.16}{#1}}
\newcommand{\ErrorTok}[1]{\textcolor[rgb]{0.64,0.00,0.00}{\textbf{#1}}}
\newcommand{\NormalTok}[1]{#1}
\usepackage{graphicx,grffile}
\makeatletter
\def\maxwidth{\ifdim\Gin@nat@width>\linewidth\linewidth\else\Gin@nat@width\fi}
\def\maxheight{\ifdim\Gin@nat@height>\textheight\textheight\else\Gin@nat@height\fi}
\makeatother
% Scale images if necessary, so that they will not overflow the page
% margins by default, and it is still possible to overwrite the defaults
% using explicit options in \includegraphics[width, height, ...]{}
\setkeys{Gin}{width=\maxwidth,height=\maxheight,keepaspectratio}
\IfFileExists{parskip.sty}{%
\usepackage{parskip}
}{% else
\setlength{\parindent}{0pt}
\setlength{\parskip}{6pt plus 2pt minus 1pt}
}
\setlength{\emergencystretch}{3em}  % prevent overfull lines
\providecommand{\tightlist}{%
  \setlength{\itemsep}{0pt}\setlength{\parskip}{0pt}}
\setcounter{secnumdepth}{0}
% Redefines (sub)paragraphs to behave more like sections
\ifx\paragraph\undefined\else
\let\oldparagraph\paragraph
\renewcommand{\paragraph}[1]{\oldparagraph{#1}\mbox{}}
\fi
\ifx\subparagraph\undefined\else
\let\oldsubparagraph\subparagraph
\renewcommand{\subparagraph}[1]{\oldsubparagraph{#1}\mbox{}}
\fi

%%% Use protect on footnotes to avoid problems with footnotes in titles
\let\rmarkdownfootnote\footnote%
\def\footnote{\protect\rmarkdownfootnote}

%%% Change title format to be more compact
\usepackage{titling}

% Create subtitle command for use in maketitle
\newcommand{\subtitle}[1]{
  \posttitle{
    \begin{center}\large#1\end{center}
    }
}

\setlength{\droptitle}{-2em}

  \title{R Notebook}
    \pretitle{\vspace{\droptitle}\centering\huge}
  \posttitle{\par}
    \author{}
    \preauthor{}\postauthor{}
    \date{}
    \predate{}\postdate{}
  

\begin{document}
\maketitle

Sea Ice Records

\begin{Shaded}
\begin{Highlighting}[]
\NormalTok{ARC <-}\StringTok{ }\KeywordTok{read.csv}\NormalTok{(}\StringTok{"ftp://sidads.colorado.edu/DATASETS/NOAA/G02135/north/daily/data/N_seaice_extent_daily_v3.0.csv"}\NormalTok{, }\DataTypeTok{header =} \OtherTok{FALSE}\NormalTok{, }\DataTypeTok{skip =} \DecValTok{2}\NormalTok{)}

\NormalTok{ARC <-}\StringTok{ }\NormalTok{ARC[,}\DecValTok{1}\OperatorTok{:}\DecValTok{4}\NormalTok{]}
\end{Highlighting}
\end{Shaded}

paste the Year, Month, Day columns and utrn them into a date that R can
recognize

\begin{Shaded}
\begin{Highlighting}[]
\NormalTok{ARC_tm <-}\StringTok{ }\KeywordTok{paste}\NormalTok{(ARC[,}\DecValTok{1}\NormalTok{], ARC[,}\DecValTok{2}\NormalTok{], ARC[,}\DecValTok{3}\NormalTok{], }\DataTypeTok{sep =} \StringTok{"-"}\NormalTok{)}
\NormalTok{ARC_tm <-}\StringTok{ }\KeywordTok{strptime}\NormalTok{(ARC_tm, }\DataTypeTok{format =} \StringTok{"%Y-%m-%d"}\NormalTok{, }\DataTypeTok{tz =} \StringTok{"GMT"}\NormalTok{)}
\end{Highlighting}
\end{Shaded}

Designate the 4th column as the SIE

\begin{Shaded}
\begin{Highlighting}[]
\NormalTok{ARC_SIE <-}\StringTok{ }\NormalTok{ARC[,}\DecValTok{4}\NormalTok{]}

\NormalTok{ANT <-}\StringTok{ }\KeywordTok{read.csv}\NormalTok{(}\StringTok{"ftp://sidads.colorado.edu/DATASETS/NOAA/G02135/south/daily/data/S_seaice_extent_daily_v3.0.csv"}\NormalTok{, }\DataTypeTok{header =} \OtherTok{FALSE}\NormalTok{, }\DataTypeTok{skip =} \DecValTok{2}\NormalTok{)}
\NormalTok{ANT <-}\StringTok{ }\NormalTok{ANT[,}\DecValTok{1}\OperatorTok{:}\DecValTok{4}\NormalTok{]}
\NormalTok{ANT_tm <-}\StringTok{ }\KeywordTok{paste}\NormalTok{(ANT[,}\DecValTok{1}\NormalTok{], ANT[,}\DecValTok{2}\NormalTok{], ANT[,}\DecValTok{3}\NormalTok{], }\DataTypeTok{sep =} \StringTok{"-"}\NormalTok{)}
\NormalTok{ANT_tm <-}\StringTok{ }\KeywordTok{strptime}\NormalTok{(ANT_tm, }\DataTypeTok{format =} \StringTok{"%Y-%m-%d"}\NormalTok{, }\DataTypeTok{tz =} \StringTok{"GMT"}\NormalTok{)}
\end{Highlighting}
\end{Shaded}

Designate the 4th column as the SIE

\begin{Shaded}
\begin{Highlighting}[]
\NormalTok{ANT_SIE <-}\StringTok{ }\NormalTok{ANT[,}\DecValTok{4}\NormalTok{]}
\end{Highlighting}
\end{Shaded}

Plot the record of Arctic and Antarctic sea ice extent from 1979 to
present

\begin{Shaded}
\begin{Highlighting}[]
\KeywordTok{plot}\NormalTok{(}\DataTypeTok{x=}\NormalTok{ ANT_tm, }\DataTypeTok{y=}\NormalTok{ANT_SIE, }\DataTypeTok{type =} \StringTok{"l"}\NormalTok{, }\DataTypeTok{lwd =} \DecValTok{2}\NormalTok{, }\DataTypeTok{col =} \StringTok{"grey"}\NormalTok{, }\DataTypeTok{xlab =} \StringTok{"Date"}\NormalTok{, }\DataTypeTok{ylab =} \StringTok{"Sea Ice Extent"}\NormalTok{, }\DataTypeTok{main =} \StringTok{"Daily Sea Ice Extent (millions sq. km)"}\NormalTok{ , }\DataTypeTok{ylim =} \KeywordTok{c}\NormalTok{(}\DecValTok{0}\NormalTok{, }\DecValTok{20}\NormalTok{))}

\KeywordTok{lines}\NormalTok{(}\DataTypeTok{x =}\NormalTok{ ARC_tm, }\DataTypeTok{y =}\NormalTok{ARC_SIE, }\DataTypeTok{type =} \StringTok{"l"}\NormalTok{, }\DataTypeTok{lwd =} \DecValTok{2}\NormalTok{, }\DataTypeTok{col =} \StringTok{"black"}\NormalTok{)}
\KeywordTok{legend}\NormalTok{(}\DataTypeTok{x =} \StringTok{"bottomright"}\NormalTok{, }\DataTypeTok{legend =} \KeywordTok{c}\NormalTok{(}\StringTok{"Arctic"}\NormalTok{, }\StringTok{"Antarctic"}\NormalTok{), }\DataTypeTok{lwd =} \DecValTok{2}\NormalTok{, }\DataTypeTok{col =} \KeywordTok{c}\NormalTok{(}\StringTok{"black"}\NormalTok{ , }\StringTok{"grey"}\NormalTok{), }\DataTypeTok{bty =} \StringTok{"n"}\NormalTok{, }\DataTypeTok{horiz =} \OtherTok{TRUE}\NormalTok{)}
\end{Highlighting}
\end{Shaded}

\includegraphics{SIEJan2019_files/figure-latex/unnamed-chunk-5-1.pdf}

Read in the Arctic March and Sept

\begin{Shaded}
\begin{Highlighting}[]
\NormalTok{ARC_MAR <-}\StringTok{ }\KeywordTok{read.csv}\NormalTok{(}\StringTok{"ftp://sidads.colorado.edu/DATASETS/NOAA/G02135/north/monthly/data/N_03_extent_v3.0.csv"}\NormalTok{, }\DataTypeTok{header =} \OtherTok{TRUE}\NormalTok{)}
\NormalTok{ARC_SEP <-}\StringTok{ }\KeywordTok{read.csv}\NormalTok{(}\StringTok{"ftp://sidads.colorado.edu/DATASETS/NOAA/G02135/north/monthly/data/N_09_extent_v3.0.csv"}\NormalTok{, }\DataTypeTok{header =} \OtherTok{TRUE}\NormalTok{)}
\end{Highlighting}
\end{Shaded}

Read in Ant

\begin{Shaded}
\begin{Highlighting}[]
\NormalTok{ANT_MAR <-}\StringTok{ }\KeywordTok{read.csv}\NormalTok{(}\StringTok{"ftp://sidads.colorado.edu/DATASETS/NOAA/G02135/south/monthly/data/S_03_extent_v3.0.csv"}\NormalTok{, }\DataTypeTok{header =} \OtherTok{TRUE}\NormalTok{)}
\NormalTok{ANT_SEP <-}\StringTok{ }\KeywordTok{read.csv}\NormalTok{(}\StringTok{"ftp://sidads.colorado.edu/DATASETS/NOAA/G02135/south/monthly/data/S_09_extent_v3.0.csv"}\NormalTok{, }\DataTypeTok{header =} \OtherTok{TRUE}\NormalTok{)}
\end{Highlighting}
\end{Shaded}

Plot Arctic SIE

\begin{Shaded}
\begin{Highlighting}[]
\KeywordTok{plot}\NormalTok{(}\DataTypeTok{x =}\NormalTok{ ARC_MAR[,}\DecValTok{1}\NormalTok{], }\DataTypeTok{y =}\NormalTok{ ARC_MAR[,}\DecValTok{5}\NormalTok{], }\DataTypeTok{type =} \StringTok{"o"}\NormalTok{, }\DataTypeTok{lwd =} \DecValTok{2}\NormalTok{, }\DataTypeTok{xlab =} \StringTok{"Year"}\NormalTok{, }\DataTypeTok{ylab =} \StringTok{"Sea Ice Extent"}\NormalTok{, }\DataTypeTok{main =} \StringTok{"Arctic Monthly Sea Ice Extent }\CharTok{\textbackslash{}n}\StringTok{ (million sq. km.)"}\NormalTok{, }\DataTypeTok{ylim =} \KeywordTok{c}\NormalTok{(}\DecValTok{0}\NormalTok{,}\DecValTok{20}\NormalTok{), }\DataTypeTok{pch =} \DecValTok{1}\NormalTok{)}

\KeywordTok{points}\NormalTok{(}\DataTypeTok{x =}\NormalTok{ ARC_SEP[,}\DecValTok{1}\NormalTok{], }\DataTypeTok{y =}\NormalTok{ ARC_SEP[,}\DecValTok{5}\NormalTok{], }\DataTypeTok{type =} \StringTok{"o"}\NormalTok{, }\DataTypeTok{lwd =} \DecValTok{2}\NormalTok{, }\DataTypeTok{pch =} \DecValTok{20}\NormalTok{)}
\KeywordTok{legend}\NormalTok{(}\DataTypeTok{x =} \StringTok{"bottomleft"}\NormalTok{, }\DataTypeTok{legend =} \KeywordTok{c}\NormalTok{(}\StringTok{"Mar."}\NormalTok{ , }\StringTok{"Sep."}\NormalTok{), }\DataTypeTok{pch =} \KeywordTok{c}\NormalTok{(}\DecValTok{1}\NormalTok{,}\DecValTok{20}\NormalTok{), }\DataTypeTok{lwd =} \DecValTok{2}\NormalTok{, }\DataTypeTok{horiz =} \OtherTok{TRUE}\NormalTok{, }\DataTypeTok{bty =} \StringTok{"n"}\NormalTok{)}
\end{Highlighting}
\end{Shaded}

\includegraphics{SIEJan2019_files/figure-latex/unnamed-chunk-8-1.pdf}

Plot Antacrtica SIE

\begin{Shaded}
\begin{Highlighting}[]
\KeywordTok{plot}\NormalTok{(}\DataTypeTok{x =}\NormalTok{ ANT_MAR[,}\DecValTok{1}\NormalTok{], }\DataTypeTok{y =}\NormalTok{ ANT_MAR[,}\DecValTok{5}\NormalTok{], }\DataTypeTok{type =} \StringTok{"o"}\NormalTok{, }\DataTypeTok{lwd =} \DecValTok{2}\NormalTok{, }\DataTypeTok{xlab =} \StringTok{"Year"}\NormalTok{, }\DataTypeTok{ylab =} \StringTok{"Sea Ice Extent"}\NormalTok{, }\DataTypeTok{main =} \StringTok{"Antarctic Monthly Sea Ice Extent }\CharTok{\textbackslash{}n}\StringTok{ (million sq. km.)"}\NormalTok{, }\DataTypeTok{ylim =} \KeywordTok{c}\NormalTok{(}\DecValTok{0}\NormalTok{,}\DecValTok{20}\NormalTok{), }\DataTypeTok{pch =} \DecValTok{1}\NormalTok{)}

\KeywordTok{points}\NormalTok{(}\DataTypeTok{x =}\NormalTok{ ANT_SEP[,}\DecValTok{1}\NormalTok{], }\DataTypeTok{y =}\NormalTok{ ANT_SEP[,}\DecValTok{5}\NormalTok{], }\DataTypeTok{type =} \StringTok{"o"}\NormalTok{, }\DataTypeTok{lwd =} \DecValTok{2}\NormalTok{, }\DataTypeTok{pch =} \DecValTok{20}\NormalTok{)}
\KeywordTok{legend}\NormalTok{(}\DataTypeTok{x =} \StringTok{"bottomleft"}\NormalTok{, }\DataTypeTok{legend =} \KeywordTok{c}\NormalTok{(}\StringTok{"Mar."}\NormalTok{ , }\StringTok{"Sep."}\NormalTok{), }\DataTypeTok{pch =} \KeywordTok{c}\NormalTok{(}\DecValTok{1}\NormalTok{,}\DecValTok{20}\NormalTok{), }\DataTypeTok{lwd =} \DecValTok{2}\NormalTok{, }\DataTypeTok{horiz =} \OtherTok{TRUE}\NormalTok{, }\DataTypeTok{bty =} \StringTok{"n"}\NormalTok{)}
\end{Highlighting}
\end{Shaded}

\includegraphics{SIEJan2019_files/figure-latex/unnamed-chunk-9-1.pdf}


\end{document}
